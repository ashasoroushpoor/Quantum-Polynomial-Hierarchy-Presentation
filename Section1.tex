
    \section{Section 1}
    \begin{frame}{Simple frame}
        \begin{enumerate}
            \item Item 1 pause.
            \pause
            \item Item 2 pause.
            \pause
            \item Item 3 pause.
        \end{enumerate}
    \end{frame}
    \begin{frame}

    \end{frame}
    \subsection{Subsection 1.1}

    \begin{frame}{Simple frame}
        This is a simple frame.
    \end{frame}
    
    % \begin{frame}[plain]{Plain frame}
    \begin{frame}{Plain frame}
        This is a frame with plain style and it is numbered.
    \end{frame}
    \subsection{Subsection 1.2}
    % \begin{frame}[t]
    \begin{frame}
        This frame has an empty title and is aligned to top.
    \end{frame}
    
    % \begin{frame}[noframenumbering]{No frame numbering}
    \begin{frame}{No frame numbering}
        This frame is not numbered and is citing reference \cite{knuth74}.
    \end{frame}
    
    \begin{frame}{Typesetting and Math}
        The packages \texttt{fontenc} and \texttt{FiraSans}\footnote{\url{https://fonts.google.com/specimen/Fira+Sans}}\textsuperscript{,}\footnote{\url{http://mozilla.github.io/Fira/}} are used to properly set the main fonts.
        \vfill
        This theme provides styling commands to typeset \emph{emphasized}, \alert{alerted}, \textbf{bold}, \textcolor{example}{example text}, \dots
        \vfill
        \texttt{FiraSans} also provides support for mathematical symbols:
        \begin{align*}
            e^{i\pi} + 1 & = 0, \\
            \int_{-\infty}^\infty e^{-x^2}\,\mathrm{d}x & = \sqrt{\pi}.
        \end{align*}
    \end{frame}
    % \section{new section}
    % \begin{frame}

    % \end{frame}
    % \begin{frame}

    % \end{frame}
    % \begin{frame}

    % \end{frame}